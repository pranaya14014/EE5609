\documentclass[journal,12pt]{IEEEtran}

\usepackage{setspace}
\usepackage{gensymb}
\singlespacing
\usepackage[cmex10]{amsmath}
\newcommand\myemptypage{
	\null
	\thispagestyle{empty}
	\addtocounter{page}{-1}
	\newpage
}
\usepackage{amsthm}
\usepackage{mdframed}
\usepackage{mathrsfs}
\usepackage{txfonts}
\usepackage{stfloats}
\usepackage{bm}
\usepackage{cite}
\usepackage{cases}
\usepackage{subfig}

\usepackage{longtable}
\usepackage{multirow}

\usepackage{enumitem}
\usepackage{mathtools}
\usepackage{steinmetz}
\usepackage{tikz}
\usepackage{circuitikz}
\usepackage{verbatim}
\usepackage{tfrupee}
\usepackage[breaklinks=true]{hyperref}
\usepackage{graphicx}
\usepackage{tkz-euclide}

\usetikzlibrary{calc,math}
\usepackage{listings}
    \usepackage{color}                                            %%
    \usepackage{array}                                            %%
    \usepackage{longtable}                                        %%
    \usepackage{calc}                                             %%
    \usepackage{multirow}                                         %%
    \usepackage{hhline}                                           %%
    \usepackage{ifthen}                                           %%
    \usepackage{lscape}     
\usepackage{multicol}
\usepackage{chngcntr}

\DeclareMathOperator*{\Res}{Res}

\renewcommand\thesection{\arabic{section}}
\renewcommand\thesubsection{\thesection.\arabic{subsection}}
\renewcommand\thesubsubsection{\thesubsection.\arabic{subsubsection}}

\renewcommand\thesectiondis{\arabic{section}}
\renewcommand\thesubsectiondis{\thesectiondis.\arabic{subsection}}
\renewcommand\thesubsubsectiondis{\thesubsectiondis.\arabic{subsubsection}}


\hyphenation{op-tical net-works semi-conduc-tor}
\def\inputGnumericTable{}                                 %%

\lstset{
%language=C,
frame=single, 
breaklines=true,
columns=fullflexible
}
\begin{document}
\onecolumn

\newtheorem{theorem}{Theorem}[section]
\newtheorem{problem}{Problem}
\newtheorem{proposition}{Proposition}[section]
\newtheorem{lemma}{Lemma}[section]
\newtheorem{corollary}[theorem]{Corollary}
\newtheorem{example}{Example}[section]
\newtheorem{definition}[problem]{Definition}

\newcommand{\BEQA}{\begin{eqnarray}}
\newcommand{\EEQA}{\end{eqnarray}}
\newcommand{\define}{\stackrel{\triangle}{=}}
\bibliographystyle{IEEEtran}
\raggedbottom
\setlength{\parindent}{0pt}
\providecommand{\mbf}{\mathbf}
\providecommand{\pr}[1]{\ensuremath{\Pr\left(#1\right)}}
\providecommand{\qfunc}[1]{\ensuremath{Q\left(#1\right)}}
\providecommand{\sbrak}[1]{\ensuremath{{}\left[#1\right]}}
\providecommand{\lsbrak}[1]{\ensuremath{{}\left[#1\right.}}
\providecommand{\rsbrak}[1]{\ensuremath{{}\left.#1\right]}}
\providecommand{\brak}[1]{\ensuremath{\left(#1\right)}}
\providecommand{\lbrak}[1]{\ensuremath{\left(#1\right.}}
\providecommand{\rbrak}[1]{\ensuremath{\left.#1\right)}}
\providecommand{\cbrak}[1]{\ensuremath{\left\{#1\right\}}}
\providecommand{\lcbrak}[1]{\ensuremath{\left\{#1\right.}}
\providecommand{\rcbrak}[1]{\ensuremath{\left.#1\right\}}}
\theoremstyle{remark}
\newtheorem{rem}{Remark}
\newcommand{\sgn}{\mathop{\mathrm{sgn}}}
\providecommand{\abs}[1]{\left\vert#1\right\vert}
\providecommand{\res}[1]{\Res\displaylimits_{#1}} 
\providecommand{\norm}[1]{\left\lVert#1\right\rVert}
%\providecommand{\norm}[1]{\lVert#1\rVert}
\providecommand{\mtx}[1]{\mathbf{#1}}
\providecommand{\mean}[1]{E\left[ #1 \right]}
\providecommand{\fourier}{\overset{\mathcal{F}}{ \rightleftharpoons}}
%\providecommand{\hilbert}{\overset{\mathcal{H}}{ \rightleftharpoons}}
\providecommand{\system}{\overset{\mathcal{H}}{ \longleftrightarrow}}
	%\newcommand{\solution}[2]{\textbf{Solution:}{#1}}
\newcommand{\solution}{\noindent \textbf{Solution: }}
\newcommand{\cosec}{\,\text{cosec}\,}
\providecommand{\dec}[2]{\ensuremath{\overset{#1}{\underset{#2}{\gtrless}}}}
\newcommand{\myvec}[1]{\ensuremath{\begin{pmatrix}#1\end{pmatrix}}}
\newcommand{\mydet}[1]{\ensuremath{\begin{vmatrix}#1\end{vmatrix}}}
\numberwithin{equation}{subsection}
\makeatletter
\@addtoreset{figure}{problem}
\makeatother
\let\StandardTheFigure\thefigure
\let\vec\mathbf
\renewcommand{\thefigure}{\theproblem}
\def\putbox#1#2#3{\makebox[0in][l]{\makebox[#1][l]{}\raisebox{\baselineskip}[0in][0in]{\raisebox{#2}[0in][0in]{#3}}}}
     \def\rightbox#1{\makebox[0in][r]{#1}}
     \def\centbox#1{\makebox[0in]{#1}}
     \def\topbox#1{\raisebox{-\baselineskip}[0in][0in]{#1}}
     \def\midbox#1{\raisebox{-0.5\baselineskip}[0in][0in]{#1}}
\vspace{3cm}
\title{EE5609: Matrix Theory\\
          Assignment-19\\}
\author{Y.Pranaya\\
AI20MTECH14014 }
\maketitle
\bigskip
\renewcommand{\thefigure}{\theenumi}
\renewcommand{\thetable}{\theenumi}
\renewcommand{\theequation}{\arabic{equation}}
\begin{abstract}
This document contains a solution for a quadratic form of given matrix.
\end{abstract}
Download the latex-tikz codes from 

%
\begin{lstlisting}
https://github.com/pranaya14014/EE5609/blob/master/Assignment19
\end{lstlisting}
%

\section{PROBLEM}
Let
\begin{align}
\vec{A} = \myvec{1&2&0\\0&0&-2\\0&0&1}
\end{align}
and define for $x,y,z \in \mathbb{R}$
\begin{align}
\vec{Q}(x,y,z)= \myvec{x&y&z}\vec{A}\myvec{x\\y\\z}
\end{align}
Which of the following are True?
\begin{enumerate}
\item[1.] The matrix of second order partial derivatives of the quadratic form of $\vec{Q}$ is $2\vec{A}$.
\item[2.] The rank of the quadratic form of $\vec{Q}$ is 2
\item[3.] The signature of the quadratic form $\vec{Q}$ is $(++0)$
\item[4.] The quadratic form $\vec{Q}$ takes the value 0 for some non-zero vector $(x,y,z)$
\end{enumerate}
\section{SOLUTION}
\renewcommand{\thetable}{1}
\begin{longtable}{|c|l|}
    \hline
	\multirow{3}{*}{Quadratic Form of a matrix} 
	& \\
	& Let $\vec{V}$ be a vector space over $\mathbb{R}$. $\vec{A}$ be a symmetric matrix $n\times n$.\\& Quadratic form on $\vec{V}$ is a real function, $(\vec{F}:\vec{V}\rightarrow\mathbb{R})$ defined as 
	\\& $F(x)= \vec{x}\vec{A}\vec{x}^T= \sum_{i,j=1}^{n}a_{ij}x_ix_j$  where $\vec{x} \in \vec{V}$ \\

	\hline
	\multirow{3}{*}{Signature of Quadratic form} 
	& \\
	& The signature of quadratic form is $(n_{+},n_{-},n_{0})$ \\ & where $n_{+}$ is the number of positive entries, $n_{-}$ is number of negative entries and \\ & $n_{0}$ is number of zero's in $a_{ii}$ \\
	\hline
	\multirow{3}{*}{Rank of quadratic form} 
	& \\
	& Rank of quadratic form is the rank of its matrix \\
	& which is maximum number of linearly independent rows/columns of a matrix \\
	\hline
	\caption{Definitions}
    \label{Table.1}
\end{longtable}
 
\renewcommand{\thetable}{2}
\begin{table*}
\begin{tabular*}{\textwidth}{|c|@{\extracolsep{\fill}}|c|}
\hline
\textbf{Option 1} & The matrix of second order partial derivatives of the quadratic form of $\vec{Q}$ is $2\vec{A}$.\\
\hline
\multirow{3}{*}{Solution} & $\vec{Q}(x,y,z)$= \myvec{x&y&z}$\vec{A}$\myvec{x\\y\\z} = \myvec{x&y&z}\myvec{x+2y\\-2z\\z} = $x^2+z^2+2xy-2yz$ \\
& First order partial derivaties: $\frac{\partial\vec{Q}}{\partial x}= 2x+2y \quad \frac{\partial\vec{Q}}{\partial y} = 2x-2z \quad \frac{\partial\vec{Q}}{\partial z}= 2z-2y $\\
& Second order partial derivatives of: $\frac{\partial^2\vec{Q}}{\partial x^2}= 2 \quad \frac{\partial^2\vec{Q}}{\partial y^2}= 0 \quad \frac{\partial^2\vec{Q}}{\partial z^2}= 2$\\
&$ \quad \frac{\partial^2 \vec{Q}}{\partial x \partial y}= \frac{\partial^2 \vec{Q}}{\partial y \partial x}= 2 \quad  \frac{\partial^2 \vec{Q}}{\partial x \partial z}= \frac{\partial^2 \vec{Q}}{\partial z \partial x}= 0 \quad \frac{\partial^2 \vec{Q}}{\partial y \partial z}= \frac{\partial^2 \vec{Q}}{\partial z \partial y}= -2$  \\
& Matrix of second order partial derivatives $\vec{Q}$: \myvec{\frac{\partial^2\vec{Q}}{\partial x^2} &\frac{\partial^2 \vec{Q}}{\partial x \partial y}& \frac{\partial^2 \vec{Q}}{\partial x \partial z} \\ \frac{\partial^2 \vec{Q}}{\partial y \partial x} & \frac{\partial^2\vec{Q}}{\partial y^2}&  \frac{\partial^2 \vec{Q}}{\partial y \partial z}\\ \frac{\partial^2 \vec{Q}}{\partial z \partial x} & \frac{\partial^2 \vec{Q}}{\partial z \partial y} &\frac{\partial^2\vec{Q}}{\partial z^2}} = \myvec{2&2&0\\2&0&-2\\0&-2&2} $\neq 2\vec{A}$\\
& Hence, $\textbf{Option 1}$ is not correct.\\
\hline
\end{tabular*}
\label{Table.2}
\caption{Solution for Option 1}
\end{table*}
\renewcommand{\thetable}{3}
\begin{table*}
\begin{tabular*}{\textwidth}{|c|@{\extracolsep{\fill}}|c|}
\hline
\textbf{Option 2} & The rank of the quadratic form of $\vec{Q}$ is 2\\
\hline
\multirow{3}{*}{Solution} & From above we have quadratic form of $\vec{Q}=\myvec{2&2&0\\2&0&-2\\0&-2&2} $\\
& Echelon form reduction: \myvec{2&2&0\\2&0&-2\\0&-2&2}$\xleftrightarrow{R_1=\frac{1}{2}}$\myvec{1&1&0\\2&0&-2\\0&-2&2}
 $\xleftrightarrow{R_2 \rightarrow R_2-2R_1}$ \myvec{1&1&0\\0&-2&-2\\0&-2&2} \\& $\xleftrightarrow{R_2 \rightarrow \frac{-1}{2}R_2}$ \myvec{1&1&0\\0&1&1\\0&-2&2} $\xleftrightarrow{R_3 \rightarrow R_3+2R_2}$ \myvec{1&1&0\\0&1&1\\0&0&4} $\xleftrightarrow{R_3 \rightarrow \frac{1}{4}R_3}$ \myvec{1&1&0\\0&1&1\\0&0&1}
\\& $\xleftrightarrow{R_1 \rightarrow R_1-R_2}$ \myvec{1&0&0\\0&1&1\\0&0&1}$\xleftrightarrow{R_2 \rightarrow R_2-R_3}$ \myvec{1&0&0\\0&1&0\\0&0&1} \\
& Rank = Number of non-zero rows = 3 $\neq$ 2 \\
&  Hence, $\textbf{Option 2}$ is not correct.\\
\hline
\end{tabular*}
\label{Table.3}
\caption{Solution for Option 2}
\end{table*}
\renewcommand{\thetable}{4}
\begin{table*}
\begin{tabular*}{\textwidth}{|c|@{\extracolsep{\fill}}|c|}
\hline
\textbf{Option 3} & The signature of the quadratic form $\vec{Q}$ is $(++0)$\\
\hline
\multirow{3}{*}{Solution} & From above we have quadratic form of $\vec{Q}$ = \myvec{2&2&0\\2&0&-2\\0&-2&2}\\
& Finding eigen values: \mydet{\vec{Q}-\lambda\vec{I}}= \myvec{2-\lambda& 2&0\\2&-\lambda&-2\\0&-2&2-\lambda}\\&
$\implies (2-\lambda)\brak{-2\lambda +\lambda^2+ 4} +8 = 0$\\&
$\implies \lambda^3-4\lambda^2-4\lambda+16=0 $ \\&
$\lambda_1 = 4 \quad \lambda_2= 2 \quad \lambda_3 = -2$ \\&
Signature = $(n_{+},n_{-},n_{0}) = (2,1,0)\neq (++0)$\\&
Hence, $\textbf{Option 3}$ is not correct.\\
\hline
\end{tabular*}
\label{Table.4}
\caption{Solution for Option 3}
\end{table*}
\renewcommand{\thetable}{5}
\begin{table*}
\begin{tabular*}{\textwidth}{|c|@{\extracolsep{\fill}}|c|}
\hline
\textbf{Option 4} &The quadratic form $\vec{Q}$ takes the value 0 for some non-zero vector $(x,y,z)$\\
\hline
\multirow{3}{*}{Solution} & From above we have quadratic form of $\vec{Q}$ = \myvec{2&2&0\\2&0&-2\\0&-2&2}\\ &
we can see that few elements are zero even though the vectors are non-zero. \\ & Therefore, \textbf{Option 4} is correct.\\
\hline
\end{tabular*}
\label{Table.5}
\caption{Solution for Option 4}
\end{table*}

  
\end{document}